%%%%%%%%%%%%%%%%%%%%%%%%%%%%%%%%%%%%%%%%%%%%%%%%%%%%%%%%%%%%%%%
%
% Welcome to Overleaf --- just edit your LaTeX on the left,
% and we'll compile it for you on the right. If you open the
% 'Share' menu, you can invite other users to edit at the same
% time. See www.overleaf.com/learn for more info. Enjoy!
%
% Note: you can export the pdf to see the result at full
% resolution.
%
%%%%%%%%%%%%%%%%%%%%%%%%%%%%%%%%%%%%%%%%%%%%%%%%%%%%%%%%%%%%%%%
% Author: Marco Miani

\documentclass{article}
\usepackage{tikz}

%%%<
\usepackage{verbatim}
\usepackage[active,tightpage]{preview}
\PreviewEnvironment{tikzpicture}
\setlength\PreviewBorder{5pt}%
%%%>



\begin{comment}
:Title: Spherical and cartesian grids

Representation of spherical (red) and cartesian (black) computational grids used 
by SWAN_. Latter gives an example of unstructured grids. Conversion from former
to latter involves a deformation factor which is acceptable within a given spatial limit.

The drawing is based on Tomas M. Trzeciak's 
`Stereographic and cylindrical map projections example`__.

.. __: http://www.texample.net/tikz/examples/map-projections/
.. _SWAN: http://www.texample.net/tikz/examples/swan-wave-model/

\end{comment}

\usetikzlibrary{positioning}
%% helper macros

% The 3D code is based on The drawing is based on Tomas M. Trzeciak's 
% `Stereographic and cylindrical map projections example`: 
% http://www.texample.net/tikz/examples/map-projections/
\newcommand\pgfmathsinandcos[3]{%
  \pgfmathsetmacro#1{sin(#3)}%
  \pgfmathsetmacro#2{cos(#3)}%
}
\newcommand\LongitudePlane[3][current plane]{%
  \pgfmathsinandcos\sinEl\cosEl{#2} % elevation
  \pgfmathsinandcos\sint\cost{#3} % azimuth
  \tikzset{#1/.style={cm={\cost,\sint*\sinEl,0,\cosEl,(0,0)}}}
}
\newcommand\LatitudePlane[3][current plane]{%
  \pgfmathsinandcos\sinEl\cosEl{#2} % elevation
  \pgfmathsinandcos\sint\cost{#3} % latitude
  \pgfmathsetmacro\yshift{\cosEl*\sint}
  \tikzset{#1/.style={cm={\cost,0,0,\cost*\sinEl,(0,\yshift)}}} %
}
\newcommand\DrawLongitudeCircle[2][1]{
  \LongitudePlane{\angEl}{#2}
  \tikzset{current plane/.prefix style={scale=#1}}
   % angle of "visibility"
  \pgfmathsetmacro\angVis{atan(sin(#2)*cos(\angEl)/sin(\angEl))} %
  \draw[current plane,thin,black] (\angVis:1) arc (\angVis:\angVis+180:1);
  \draw[current plane,thin,dashed] (\angVis-180:1) arc (\angVis-180:\angVis:1);
}%this is fake: for drawing the grid
\newcommand\DrawLongitudeCirclered[2][1]{
  \LongitudePlane{\angEl}{#2}
  \tikzset{current plane/.prefix style={scale=#1}}
   % angle of "visibility"
  \pgfmathsetmacro\angVis{atan(sin(#2)*cos(\angEl)/sin(\angEl))} %
  \draw[current plane,red,thick] (150:1) arc (150:180:1);
  %\draw[current plane,dashed] (-50:1) arc (-50:-35:1);
}%for drawing the grid
\newcommand\DLongredd[2][1]{
  \LongitudePlane{\angEl}{#2}
  \tikzset{current plane/.prefix style={scale=#1}}
   % angle of "visibility"
  \pgfmathsetmacro\angVis{atan(sin(#2)*cos(\angEl)/sin(\angEl))} %
  \draw[current plane,black,dashed, ultra thick] (150:1) arc (150:180:1);
}
\newcommand\DLatred[2][1]{
  \LatitudePlane{\angEl}{#2}
  \tikzset{current plane/.prefix style={scale=#1}}
  \pgfmathsetmacro\sinVis{sin(#2)/cos(#2)*sin(\angEl)/cos(\angEl)}
  % angle of "visibility"
  \pgfmathsetmacro\angVis{asin(min(1,max(\sinVis,-1)))}
  \draw[current plane,dashed,black,ultra thick] (-50:1) arc (-50:-35:1);

}
\newcommand\fillred[2][1]{
  \LongitudePlane{\angEl}{#2}
  \tikzset{current plane/.prefix style={scale=#1}}
   % angle of "visibility"
  \pgfmathsetmacro\angVis{atan(sin(#2)*cos(\angEl)/sin(\angEl))} %
  \draw[current plane,red,thin] (\angVis:1) arc (\angVis:\angVis+180:1);

}

\newcommand\DrawLatitudeCircle[2][1]{
  \LatitudePlane{\angEl}{#2}
  \tikzset{current plane/.prefix style={scale=#1}}
  \pgfmathsetmacro\sinVis{sin(#2)/cos(#2)*sin(\angEl)/cos(\angEl)}
  % angle of "visibility"
  \pgfmathsetmacro\angVis{asin(min(1,max(\sinVis,-1)))}
  \draw[current plane,thin,black] (\angVis:1) arc (\angVis:-\angVis-180:1);
  \draw[current plane,thin,dashed] (180-\angVis:1) arc (180-\angVis:\angVis:1);
}%Defining functions to draw limited latitude circles (for the red mesh)
\newcommand\DrawLatitudeCirclered[2][1]{
  \LatitudePlane{\angEl}{#2}
  \tikzset{current plane/.prefix style={scale=#1}}
  \pgfmathsetmacro\sinVis{sin(#2)/cos(#2)*sin(\angEl)/cos(\angEl)}
  % angle of "visibility"
  \pgfmathsetmacro\angVis{asin(min(1,max(\sinVis,-1)))}
  %\draw[current plane,red,thick] (-\angVis-50:1) arc (-\angVis-50:-\angVis-20:1);
\draw[current plane,red,thick] (-50:1) arc (-50:-35:1);

}

\tikzset{%
  >=latex,
  inner sep=0pt,%
  outer sep=2pt,%
  mark coordinate/.style={inner sep=0pt,outer sep=0pt,minimum size=3pt,
    fill=black,circle}%
}
\usepackage{amsmath}
\usetikzlibrary{arrows}
\pagestyle{empty}
\usepackage{pgfplots}
\usetikzlibrary{calc,fadings,decorations.pathreplacing}

\begin{document}
\begin{figure}[ht!]
	\begin{tikzpicture}[scale=1,every node/.style={minimum size=1cm}]
	%% some definitions
	
	\def\R{4} % sphere radius
	
	\def\angEl{25} % elevation angle
	\def\angAz{-100} % azimuth angle
	\def\angPhiOne{-50} % longitude of point P
	\def\angPhiTwo{-35} % longitude of point Q
	\def\angBeta{30} % latitude of point P and Q
	
	%% working planes
	
	\pgfmathsetmacro\H{\R*cos(\angEl)} % distance to north pole
	\LongitudePlane[xzplane]{\angEl}{\angAz}
	\LongitudePlane[pzplane]{\angEl}{\angPhiOne}
	\LongitudePlane[qzplane]{\angEl}{\angPhiTwo}
	\LatitudePlane[equator]{\angEl}{0}
	\fill[ball color=white!10] (0,0) circle (\R); % 3D lighting effect
	\coordinate (O) at (0,0);
	\coordinate[mark coordinate] (N) at (0,\H);
	\coordinate[mark coordinate] (S) at (0,-\H);
	\path[xzplane] (\R,0) coordinate (XE);
	

	\path[qzplane] (\angBeta:\R+5.2376) coordinate (XEd);
	\path[pzplane] (\angBeta:\R) coordinate (P);
	\path[pzplane] (\angBeta:\R+5.2376) coordinate (Pd);%sfora di una 
	\path[pzplane] (\angBeta:\R+5.2376) coordinate (Td);%sfora di una 
	\path[pzplane] (\R,0) coordinate (PE);
    \path[pzplane] (\R+1,0) coordinate (PEd);
	\path[qzplane] (\angBeta:\R) coordinate (Q);
	\path[qzplane] (\angBeta:\R) coordinate (Qd);%sfora di una quantità 
	\path[qzplane] (\R,0) coordinate (QE);
    \path[qzplane] (\R+4,0) coordinate (QEd);
    

    \DrawLatitudeCircle[\R]{0} % equator
	\node[above=8pt] at (N) {$\mathbf{N}$};
	\node[below=8pt] at (S) {$\mathbf{S}$};
	
	\draw[-,dashed, thick] (N) -- (S);

    \DrawLongitudeCircle[\R]{\angPhiOne}
    \DrawLongitudeCircle[\R]{\angPhiTwo}
    \DrawLatitudeCircle[\R]{\angBeta}
    

    
%     \foreach \t in {0,2,...,30} { \DrawLatitudeCirclered[\R]{\t} }
% 	\foreach \t in {130,133,...,145} { \DrawLongitudeCirclered[\R]{\t} }


    \coordinate(brs) at (5.7,-1.65);
    \coordinate(bre) at (3.3,-1);
    \draw[-,dashed,black,thick] (brs) -- (bre);
    
    \coordinate(bls) at (4.5,-2.3);
    \coordinate(ble) at (2.54,-1.28);
    \draw[-,dashed,black,thick] (bls) -- (ble);

    

 	\foreach \t in {130,145,...,145} { \DLongredd[\R+3]{\t} }
 	\foreach \t in {130,133,...,145} { \DrawLongitudeCirclered[\R+3]{\t} }
 	\foreach \t in {0,30,...,30} { \DLatred[\R+3]{\t} }
    \foreach \t in {0,2,...,30} { \DrawLatitudeCirclered[\R+3]{\t} }
    
    \coordinate(tls) at (2.2,0.7);
    \coordinate(tle) at (4,1.2);
    \draw[-,dashed,black,thick] (tls) -- (tle);
    
    \coordinate(trs) at (4.95,1.7);
    \coordinate(tre) at (2.8,0.96);
    \draw[-,dashed,black,thick] (trs) -- (tre);
    

	
    	
\end{tikzpicture}

\end{figure}



\end{document}